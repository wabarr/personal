%% For publications, include Citation, with period afterward. Then include DOI if available, with no period after DOI

\item Geraads D, {\bfseries Barr WA}, Reed DN, Laurin M, Alemseged Z. New remains of \emph{Camelus grattardi} (Mammalia, Camelidae) from the Plio-Pleistocene of Ethiopia and the phylogeny of the genus. Journal of Mammalian Evolution. In Press.

\item Tóth, AB, Lyons SK, {\bfseries Barr WA}, Behrensmeyer AK, Blois JL, Bobe R, Davis M, Du A, Eronen J, Faith JT, Fraser D, Gotelli NJ, Graves GR, Jukar AM, Miller JH, Pineda-Munoz S, Soul LC, Villaseñor A, Alroy J. 2019. Reorganization of surviving mammal communities after the end-Pleistocene megafaunal extinction. \emph{Science.} 365:1305-1308. \href{https://dx.doi.org/10.1126/science.aaw1605 }{doi:10.1126/science.aaw1605}

\item Patterson DB, Braun DR, Allen K, {\bfseries Barr WA}, Behrensmeyer AK, Biernat M, Lehmann SB, Maddox T, Manthi FK, Merritt SR, Morris SE, O'Brien K, Reeves JS, Wood BA, Bobe R. 2019. Comparative isotopic evidence from East Turkana is consistent with a dietary shift between early \emph{Homo} and \emph{Homo erectus}. \emph{Nature Ecology and Evolution}. 3:1048-1056. \href{https:://dx.doi.org/10.1038/s41559-019-0916-0}{doi:10.1038/s41559-019-0916-0}

\item {\bfseries Barr WA}. \emph{Ecomorphology}. 2018. In: Croft DA, Simpson SW, and Su DF (eds.), \emph{Methods in Paleoecology: Reconstructing Cenozoic Terrestrial Environments and Ecological Communities}. Springer (Vertebrate Paleobiology and Paleoanthropology Series), Cham, Switzerland.  339-349. \href{https://doi.org/10.1007/978-3-319-94265-0}{doi:10.1007/978-3-319-94265-0}

\item Fraser D, Haupt R,  {\bfseries Barr WA}. 2018. Phylogenetic signal in tooth wear dietary niche proxies: What it means for those in the field. \emph{Ecology and Evolution.} \href{https://dx.doi.org/10.1002/ece3.4540}{doi:10.1002/ece3.4540}

\item {\bfseries Barr WA}. 2018. The morphology of the bovid calcaneus: function, phylogenetic signal, and allometric scaling. \emph{Journal of Mammalian Evolution.} \href{https://dx.doi.org/10.1007/s10914-018-9446-9}{doi:10.1007/s10914-018-9446-9}

\item  Reed, DN, {\bfseries Barr WA}, Kappelman J. 2018. PaleoCore: an open-source platform for geospatial data integration in paleoanthropology. In: Anemone R, Conroy G (eds.), \emph{New Geospatial Approaches in Anthropology}. University of New Mexico Press. Albuquerque, NM.

\item  Fraser D, Haupt R,  {\bfseries Barr WA}. 2018. Phylogenetic Signal In Tooth Wear Dietary Niche Proxies. \emph{Ecology and Evolution}. 8:5355-5368 \href{https://doi.org/10.1002/ece3.4052}{doi:10.1002/ece3.4052}

\item  Blondel C, Rowan J, Merceron G, Bibi F,  Negash E, {\bfseries Barr WA}, Boisserie JR. 2018. Feeding ecology of Tragelaphini (Bovidae) from the Shungura Formation, Omo Valley, Ethiopia: contribution of dental wear analyses.  \emph{Palaeogeography, Palaeoclimatology, Palaeoecology}. 496:103-120. \href{https://doi.org/10.1016/j.palaeo.2018.01.027}{doi:10.1016/j.palaeo.2018.01.027}

\item  {\bfseries Barr WA}. 2017. Signal or noise? A null model method for testing hypotheses about pulsed faunal turnover. \emph{Paleobiology}. 43:656-666. \href{https://doi.org/10.1017/pab.2017.21}{doi:10.1017/pab.2017.21}

\item  {\bfseries Barr WA}. 2017. Bovid locomotor functional trait distributions reflect land cover and annual precipitation in sub-Saharan Africa. \emph{Evolutionary Ecology Research}.  \href{http://www.evolutionary-ecology.com/issues/v18/n03/ddar3051.pdf}{18:253-269}.

\item  {\bfseries Barr WA}. 2015. Paleoenvironments of the Shungura Formation (Plio-Pleistocene: Ethiopia) based on ecomorphology of the bovid astragalus. \emph{Journal of Human Evolution}. 88:97-107. \href{http://dx.doi.org/10.1016/j.jhevol.2015.05.002}{doi:10.1016/j.jhevol.2015.05.002}

\item  Reed D, {\bfseries Barr WA}, McPherron S, Bobe R, Geraads D, Wynn J, Alemseged Z. 2015. Digital Data Collection in Paleoanthropology. \emph{Evolutionary Anthropology}. 24:238-249. \href{http://dx.doi.org/10.1002/evan.21466}{doi:10.1002/evan.21466}

\item  Thompson JC, McPherron S, Bobe R, Reed DN, {\bfseries Barr WA}, Wynn J, Marean CW, Geraads D, Alemseged Z. 2015. Taphonomy of fossils from the hominin-bearing deposits at Dikika, Ethiopia. \emph{Journal of Human Evolution}. 86:112-135. \href{http://dx.doi.org/10.1016/j.jhevol.2015.06.013}{doi:10.1016/j.jhevol.2015.06.013}

\item  {\bfseries Barr WA}. 2014 Functional Morphology of the Bovid Astragalus In Relation To Habitat: Controlling Phylogenetic Signal In Ecomorphology. \emph{Journal of Morphology}. 275:1201-1216. \href{http://dx.doi.org/10.1002/jmor.20279}{doi:10.1002/jmor.20279}

\item  {\bfseries Barr WA} and Scott RS. 2014 Phylogenetic comparative methods complement discriminant function analysis in ecomorphology. \emph{American Journal of Physical Anthropology}. 153:663-674. \href{http://dx.doi.org/10.1002/ajpa.22462}{doi:10.1002/ajpa.22462}

\item  Scott RS and {\bfseries Barr WA}. 2014. Ecomorphology and phylogenetic risk: implications for habitat reconstruction using fossil bovids. \emph{Journal of Human Evolution}. 73:47-57. \href{http://dx.doi.org/10.1016/j.jhevol.2014.02.023}{doi:10.1016/j.jhevol.2014.02.023}

\item  Reed DN, and {\bfseries Barr WA}. 2010. A preliminary account of the rodents from Pleistocene levels at Grotte des Contrebandiers (Smuggler's Cave), Morocco. \emph{Historical Biology}. 22:286-294. \href{http://dx.doi.org/10.1080/08912960903562192}{doi:10.1080/08912960903562192}




