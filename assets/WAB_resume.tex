\documentclass{article}
\usepackage[top=0.85in, bottom=0.9in, left=0.9in, right=0.9in]{geometry}
\usepackage{mdwlist}
\usepackage{enumitem}
\usepackage{longtable}
\usepackage{verbatim}
\usepackage{fancyhdr}
\usepackage[colorlinks=True, urlcolor=cyan]{hyperref}
\usepackage{helvet}
\renewcommand{\familydefault}{\sfdefault}
\pagestyle{fancy}
\pagenumbering{arabic}
\lhead{\itshape W. Andrew Barr, Ph.D.}
\chead{}
\rhead{\thepage}
\lfoot{}
\cfoot{Updated \today}
\rfoot{}

\thispagestyle{empty}

%custom environment to control hanging indent in description
\newenvironment{mylist}
{\begin{description}[style=unboxed,leftmargin=1.3cm]}
{\end{description}}


\begin{document}
\begin{center}
\noindent{\bfseries{\Huge W. Andrew Barr}}
\end{center}

\vspace{15pt}

\noindent\begin{minipage}{.60\textwidth}
\begin{flushleft}
Center for the Advanced Study of Human Paleobiology\\
Department of Anthropology\\
The George Washington University\\
\end{flushleft}
\end{minipage}
\begin{minipage}{.395\textwidth}
\begin{flushright}
800 22nd St NW, Suite 6000\\
Washington, DC 20052 \\
wabarr@gmail.com - (202) 994-3213\\
\end{flushright}
\end{minipage}


\noindent\rule[-2mm]{\textwidth}{1pt}

\section*{Education}
\begin{tabular}{p{.1\textwidth}p{0.86\textwidth}}
2014 & University of Texas at Austin. Ph.D., Anthropology. \emph{The Paleoenvironments of Early Hominins in the Omo Shungura Formation (Plio-Pleistocene, Ethiopia): Synthesizing Multiple Lines of Evidence Using Phylogenetic Ecomorphology.}\\[4pt]
2008 & University of Texas at Austin. M.A., Anthropology. \\[4pt]
2005 & Tulane University. B.S., Anthropology and French.\\
\end{tabular}

\section*{Employment Experience}
\begin{tabular}{p{.15\textwidth}p{0.8\textwidth}}
2016 - Present & Visiting Assistant Professor. Center for the Advanced Study of Human Paleobiology. Department of Anthropology. The George Washington University.\\[4pt]
2014 - 2016 & Postdoctoral Scientist. Center for the Advanced Study of Human Paleobiology. Department of Anthropology. The George Washington University. Advisor: Bernard Wood.\\[4pt]
2006 - 2014 & Teaching Assistant and Graduate Research Assistant. The University of Texas at Austin.

\end{tabular}

\section*{Fieldwork Experience}
\begin{longtable}{p{.16\textwidth}p{0.80\textwidth}}
2014 - Present & Mille-Logya Research Project, Afar Region, Ethiopia (Plio-Pleistocene). My fieldwork at MLP seeks to discover new evidence of the morphology and environmental context of earliest genus \emph{Homo}  \\[4pt]

2010 - Present & Dikika Research Project, Afar Region, Ethiopia. My fieldwork at Dikika aims to understand paleoenvironmental change through the Hadar Formation, and to document the frequency and nature of bone surface modifications at this site which preserves the oldest reported evidence of stone-tool modified bones \\[4pt]

2016 - Present & Koobi-Fora Research and Training Program, East Turkana, Kenya. My fieldwork seeks to document the variability of hominin habitats in the Turkana Basin from the Pliocene through the Pleistocene, and to understand how this variability influenced human behavior and evolution. \\[4pt]

2013 - 2014 & Great Divide Basin Project, Wyoming. Collected primate and mammalian fossils from Eocene sediments, and prospected for new localities. \\[4pt]

2007, 2008, 2010 & Dalquest Research Site, Big Bend Region, Texas. Surface collected primate and mammalian fossils in the Devil's Graveyard Formation. (Eocene: Late Uintan).\\[4pt]

2009 & Contrebandiers Cave, Temara, Morocco.  Excavated site preserving Middle Stone Age archaeology (Aterian) and hominin remains. Performed systematic analysis of rodent fauna.\\
\end{longtable}

\section*{Collaborative Research Affiliations}
\begin{tabular}{p{.16\textwidth}p{0.8\textwidth}}

2015 - Present & External Member. Evolution of Terrestrial Ecosystems Working Group. National Museum of Natural History.\\[4pt]
2014 - Present & Research Associate. Department of Paleobiology.  National Museum of Natural History.\\[4pt]
2012 - Present & Research Associate and Software Developer. PaleoCore Project.\\

\end{tabular}

\section*{Research Grants and Fellowships}
\begin{tabular}{p{.1\textwidth}p{0.86\textwidth}}
In Review & Collaborative Proposal: IRES Track I: Biological and behavioral adaptations to ecological variability through time in the Turkana Basin (Kenya). Role: Co-PI. \$276,089.\\[4pt]
%2015 & National Science Foundation (Archaeology) - Middle Pleistocene Hominin Behavior and Paleoecology at Farre, Chalbi Basin, Northern Kenya. Role: Senior Scientist. \$60,000.\\[4pt]
2013 & Named Continuing Fellowship - University of Texas at Austin Graduate School. \$29,000.\\[4pt]
2012  & Dissertation Fieldwork Grant - Wenner-Gren Foundation. \$13,317.\\[4pt]
2007 &  Graduate Research Fellowship - National Science Foundation. \$90,000.\\[4pt]
2007 &  Liberal Arts Graduate Research Fellowship - University of Texas at Austin. \$1,800.\\[4pt]
%2007, 2008 &  David Bruton, Jr. Graduate Fellowship - University of Texas at Austin. \$1,000.\\[4pt]

\end{tabular}


%% For publications, include Citation, with period afterward. Then include DOI if available, with no period after DOI

\section*{Peer-Reviewed Publications}

\begin{longtable}{p{.1\textwidth}p{0.86\textwidth}}

 In review & Fraser D, Haupt R,  {\bfseries Barr WA}. Tooth Wear Dietary Niche Proxies Show Strong Phylogenetic Signal.  In review at \emph{Ecology and Evolution}.\\[4pt]

%% In review & Alemseged Z, Wynn J, Geraads D, Reed D, {\bfseries Barr WA}, Bobe R, McPherron S, Deino A, Araya M, Sier M, Roman D. Mille-Logya: a new hominin bearing late Pliocene / early Pleistocene site in the Afar, Ethiopia. \\[4pt]

%In review & {\bfseries Barr WA}, Du A, Toth A, Jukar A,  Bercovici A,  Dommain A,  Dumouchel L,  Villasenor A,  Negash E, Behrensmeyer AK, and Lyons SK. Evaluating climatic and mid-domain effects on mammalian species richness at global and continental scales. In review at \emph{Journal of Mammology}.\\[4pt]

%% In review & Kemp A and {\bfseries Barr WA}. Rates of homoplasy vary among regions of the mammalian skeleton. In review at \emph{Journal of Human Evolution.}\\[4pt]

In press & Blondel C, Rowan J, Merceron G, Bibi F,  Negash E, {\bfseries Barr WA}, Boisserie JR. Feeding ecology of Tragelaphini (Bovidae) from the Shungura Formation, Omo Valley, Ethiopia: contribution of dental wear analyses.  \emph{Palaeogeography, Palaeoclimatology, Palaeoecology}.\\[4pt]

In press & {\bfseries Barr WA}. \emph{Ecomorphology}. To be published in D.A. Croft, S.W. Simpson, and D.F. Su (eds.), Methods in Paleoecology: Reconstructing Cenozoic Terrestrial Environments and Ecological Communities. Springer (Vertebrate Paleobiology and Paleoanthropology Series), Cham, Switzerland.\\[4pt]

In press & Reed, DN, {\bfseries Barr WA}, Kappelman J. PaleoCore: an open-source platform for geospatial data integration in paleoanthropology. To be published in Anemone R, Conroy G (eds.), \emph{New Geospatial Approaches in Anthropology}. University of New Mexico Press. Albuquerque, NM.\\[4pt]


2017 & {\bfseries Barr WA}. Signal or noise? A null model method for testing hypotheses about pulsed faunal turnover. \emph{Paleobiology}. 43:656-666. \href{https://doi.org/10.1017/pab.2017.21}{doi:10.1017/pab.2017.21}\\[4pt]

2017 & {\bfseries Barr WA}. Bovid locomotor functional trait distributions reflect land cover and annual precipitation in sub-Saharan Africa. \emph{Evolutionary Ecology Research}.  \href{http://www.evolutionary-ecology.com/issues/v18/n03/ddar3051.pdf}{18:253-269}.\\[4pt]

2015 & {\bfseries Barr WA}. Paleoenvironments of the Shungura Formation (Plio-Pleistocene: Ethiopia) based on ecomorphology of the bovid astragalus. \emph{Journal of Human Evolution}. 88:97-107. \href{http://dx.doi.org/10.1016/j.jhevol.2015.05.002}{doi:10.1016/j.jhevol.2015.05.002}\\[4pt]

2015 & Reed D, {\bfseries Barr WA}, McPherron S, Bobe R, Geraads D, Wynn J, Alemseged Z. Digital Data Collection in Paleoanthropology. \emph{Evolutionary Anthropology}. 24:238-249. \href{http://dx.doi.org/10.1002/evan.21466}{doi:10.1002/evan.21466}\\[4pt]

2015 & Thompson JC, McPherron S, Bobe R, Reed DN, {\bfseries Barr WA}, Wynn J, Marean CW, Geraads D, Alemseged Z. Taphonomy of fossils from the hominin-bearing deposits at Dikika, Ethiopia. \emph{Journal of Human Evolution}. 86:112-135. \href{http://dx.doi.org/10.1016/j.jhevol.2015.06.013}{doi:10.1016/j.jhevol.2015.06.013}\\[4pt]

2014 & {\bfseries Barr WA}. Functional Morphology of the Bovid Astragalus In Relation To Habitat: Controlling Phylogenetic Signal In Ecomorphology. \emph{Journal
of Morphology}. 275:1201-1216. \href{http://dx.doi.org/10.1002/jmor.20279}{doi:10.1002/jmor.20279}\\[4pt]

2014 & {\bfseries Barr WA} and Scott RS. Phylogenetic comparative methods complement discriminant function analysis in ecomorphology. \emph{American Journal
of Physical Anthropology}. 153:663-674. \href{http://dx.doi.org/10.1002/ajpa.22462}{doi:10.1002/ajpa.22462}\\[4pt]

2014 & Scott RS and {\bfseries Barr WA}. Ecomorphology and phylogenetic risk: implications for habitat reconstruction using fossil bovids.
\emph{Journal of Human Evolution}. 73:47-57. \href{http://dx.doi.org/10.1016/j.jhevol.2014.02.023}{doi:10.1016/j.jhevol.2014.02.023}\\[4pt]

2010 & Reed DN, and {\bfseries Barr WA}. A preliminary account of the rodents from Pleistocene levels at Grotte des Contrebandiers (Smuggler's Cave),
Morocco. \emph{Historical Biology}. 22:286-294. \href{http://dx.doi.org/10.1080/08912960903562192}{doi:10.1080/08912960903562192}\\
\end{longtable}



\section*{Courses Taught}
\begin{mylist}

\item[]\emph{Introduction to Biological Anthropology}. ANTH 1001. Undergraduate survey of the field of biological anthropology. The George Washington University, Anthropology. Taught Fall 2016, Spring 2017.

\item[] \emph{Analytical Methods in Evolutionary Anthropology}. ANTH 6413. I designed this graduate course course covering applied statistical methods (e.g, regression, ANOVA and related techniques, categorical data analysis, resampling approaches) and the R statistical programming language. This is a required course for the Hominid Paleobiology PhD program. The George Washington University, Anthropology. Taught Spring 2015, Fall 2016.

\item[] \emph{Climate Change and Human Evolution}. ANTH 3491. I designed this upper level undergraduate course covering changes in global climate through evolutionary time and the impacts on evolution, with an emphasis on humans. The George Washington University, Anthropology. Taught Spring 2016, Spring 2017.

\item[] \emph{Public Understanding of Science}. HOMP 8302.  Graduate course in which students complete semester-long public service internships. Student projects target underserved Washington, DC-area public schools and general audiences at public museums with a goal of increasing scientific literacy and creating interest in scientific careers. The George Washington University, Anthropology. Taught Spring 2016.

\item[] \emph{GIS and Remote Sensing for Archaeology and Paleontology}. ANT 391 / GRG 396. Teaching Assistant.  University of Texas at Austin, Anthropology. 2010.

\item[] \emph{Human Variation}. ANT 394C. Teaching Assistant. University of Texas at Austin, Anthropology. 2009.

\item[] \emph{Introduction to Physical Anthropology}. ANT 301. Teaching Assistant. University of Texas at Austin, Anthropology. 2006, 2007, 2010, 2011, 2012.
\end{mylist}

\section*{Public Outreach and Science Communication}
\begin{longtable}{p{.14\textwidth}p{0.82\textwidth}}
2016 - Present & \href{http://facesoffieldwork.com}{Faces of Fieldwork} - I believe that public engagement with science increases when people understand who we are and why we do what we do. I launched this outreach website to magnify the reach of the amazing fieldwork being done by diverse early career scientists. So far, the project has highlighted the work of 83 fieldworkers (57 women, 26 men, including many masters and PhD students). In 2017, this website had 6475 distinct visitors from across the world.\\[4pt]
2017 & \emph{Survivors: What Fossils Tell Us About the Past and Future}. I engaged with members of the general public at the National Museum of National History, to answer questions about extinction, and the future of evolution in light of a changing climate. \\[4pt]

2014-2015 & Volunteer, Explore UT - University wide K-12 educational open house. Organized and implemented the activity ``Leaping Lemurs of Madagascar'' on locomotion and conservation of lemurs.\\
\end{longtable}
\end{document}
