\documentclass{article}
\usepackage[top=0.85in, bottom=0.9in, left=0.9in, right=0.9in]{geometry}
\usepackage{mdwlist}
\usepackage{enumitem}
\usepackage{longtable}
\usepackage{verbatim}
\usepackage{fancyhdr}
\usepackage[colorlinks=True, urlcolor=cyan]{hyperref}
\usepackage{helvet}
\renewcommand{\familydefault}{\sfdefault}
\pagestyle{fancy}
\pagenumbering{arabic}
\lhead{\itshape W. Andrew Barr - CV}
\chead{}
\rhead{\thepage}
\lfoot{}
\cfoot{Updated \today}
\rfoot{}

\thispagestyle{empty}

%custom environment to control hanging indent in description
\newenvironment{mylist}
{\begin{description}[style=unboxed,leftmargin=1.3cm]}
{\end{description}}


\begin{document}
\begin{center}
\noindent{\bfseries{\Huge W. Andrew Barr - Curriculum Vitae}}
\end{center}

\vspace{15pt}

\noindent\begin{minipage}{.60\textwidth}
\begin{flushleft}
Center for the Advanced Study of Human Paleobiology\\
Department of Anthropology\\
The George Washington University\\
\end{flushleft}
\end{minipage}
\begin{minipage}{.395\textwidth}
\begin{flushright}
800 22nd St NW, Suite 6000\\
Washington, DC 20052 \\
wabarr@gmail.com - (202) 994-3213\\
\end{flushright}
\end{minipage}


\noindent\rule[-2mm]{\textwidth}{1pt}

\section*{Education}

\begin{tabular}{p{.1\textwidth}p{0.86\textwidth}}
2014 & University of Texas at Austin. Ph.D., Anthropology. \emph{The Paleoenvironments of Early Hominins in the Omo Shungura Formation (Plio-Pleistocene, Ethiopia): Synthesizing Multiple Lines of Evidence Using Phylogenetic Ecomorphology.}\\[4pt]
2008 & University of Texas at Austin. M.A., Anthropology. \\[4pt]
2005 & Tulane University. B.S., Anthropology and French.\\
\end{tabular}


\section*{Academic Appointments}
\begin{tabular}{p{.15\textwidth}p{0.8\textwidth}}
2016 - Present & Visiting Assistant Professor. Center for the Advanced Study of Human Paleobiology. Department of Anthropology. The George Washington University.\\[4pt]
2014 - 2016 & Postdoctoral Scientist. Center for the Advanced Study of Human Paleobiology. Department of Anthropology. The George Washington University. Advisor: Bernard Wood.\\[4pt]
2014 - Present & Research Associate. Department of Paleobiology.  National Museum of Natural History.\\
\end{tabular}

\section*{Research Grants and Fellowships}
\begin{tabular}{p{.1\textwidth}p{0.86\textwidth}}
%2015 & National Science Foundation (Archaeology) - Middle Pleistocene Hominin Behavior and Paleoecology at Farre, Chalbi Basin, Northern Kenya. PI: Ferraro. Role: Senior Scientist. \$60,000.\\[4pt]
2013 & Named Continuing Fellowship - University of Texas at Austin Graduate School. \$29,000.\\[4pt]
2012  & Dissertation Fieldwork Grant - Wenner-Gren Foundation. \$13,317.\\[4pt]
2007 &  Graduate Research Fellowship - National Science Foundation. \$90,000.\\[4pt]
2007 &  Liberal Arts Graduate Research Fellowship - University of Texas at Austin. \$1,800.\\[4pt]
%2007, 2008 &  David Bruton, Jr. Graduate Fellowship - University of Texas at Austin. \$1,000.\\[4pt]

\end{tabular}


%% For publications, include Citation, with period afterward. Then include DOI if available, with no period after DOI

\section*{Peer-Reviewed Publications}

\begin{longtable}{p{.1\textwidth}p{0.86\textwidth}}

%% In review & Fraser D, Haupt R,  {\bfseries Barr WA}. Tooth wear shows strong phylogenetic signal. \\[4pt]

%% In review & Alemseged Z, Wynn J, Geraads D, Reed D, {\bfseries Barr WA}, Bobe R, McPherron S, Deino A, Araya M, Sier M, Roman D. Mille-Logya: a new hominin bearing late Pliocene / early Pleistocene site in the Afar, Ethiopia. \\[4pt]

In review & {\bfseries Barr WA}, Du A, Toth A, Jukar A,  Bercovici A,  Dommain A,  Dumouchel L,  Villasenor A,  Negash E, Behrensmeyer AK, and Lyons SK. Evaluating climatic and mid-domain effects on mammalian species richness at global and continental scales. In review at \emph{Journal of Mammology}.\\[4pt]

In review & Blondel C, Rowan J, Merceron G, Bibi F,  Negash E, {\bfseries Barr WA}, Boisserie JR. Feeding ecology of Tragelaphini (Bovidae) from the Shungura Formation, Omo Valley, Ethiopia: contribution of dental wear analyses. In review at \emph{Palaeogeography, Palaeoclimatology, Palaeoecology}.\\[4pt]

%% In review & Kemp A and {\bfseries Barr WA}. Rates of homoplasy vary among regions of the mammalian skeleton. In review at \emph{Journal of Human Evolution.}\\[4pt]

In press & {\bfseries Barr WA}. \emph{Ecomorphology}. To be published in D.A. Croft, S.W. Simpson, and D.F. Su (eds.), Methods in Paleoecology: Reconstructing Cenozoic Terrestrial Environments and Ecological Communities. Springer (Vertebrate Paleobiology and Paleoanthropology Series), Cham, Switzerland.\\[4pt]

In press & Reed, DN, {\bfseries Barr WA}, Kappelman J. PaleoCore: an open-source platform for geospatial data integration in paleoanthropology. To be published in Anemone R, Conroy G (eds.), \emph{New Geospatial Approaches in Anthropology}. University of New Mexico Press. Albuquerque, NM.\\[4pt]


2017 & {\bfseries Barr WA}. Signal or noise? A null model method for testing hypotheses about pulsed faunal turnover. \emph{Paleobiology}. 43:656-666. \href{https://doi.org/10.1017/pab.2017.21}{doi:10.1017/pab.2017.21}\\[4pt]

2017 & {\bfseries Barr WA}. Bovid locomotor functional trait distributions reflect land cover and annual precipitation in sub-Saharan Africa. \emph{Evolutionary Ecology Research}.  \href{http://www.evolutionary-ecology.com/issues/v18/n03/ddar3051.pdf}{18:253-269}.\\[4pt]

2015 & {\bfseries Barr WA}. Paleoenvironments of the Shungura Formation (Plio-Pleistocene: Ethiopia) based on ecomorphology of the bovid astragalus. \emph{Journal of Human Evolution}. 88:97-107. \href{http://dx.doi.org/10.1016/j.jhevol.2015.05.002}{doi:10.1016/j.jhevol.2015.05.002}\\[4pt]

2015 & Reed D, {\bfseries Barr WA}, McPherron S, Bobe R, Geraads D, Wynn J, Alemseged Z. Digital Data Collection in Paleoanthropology. \emph{Evolutionary Anthropology}. 24:238-249. \href{http://dx.doi.org/10.1002/evan.21466}{doi:10.1002/evan.21466}\\[4pt]

2015 & Thompson JC, McPherron S, Bobe R, Reed DN, {\bfseries Barr WA}, Wynn J, Marean CW, Geraads D, Alemseged Z. Taphonomy of fossils from the hominin-bearing deposits at Dikika, Ethiopia. \emph{Journal of Human Evolution}. 86:112-135. \href{http://dx.doi.org/10.1016/j.jhevol.2015.06.013}{doi:10.1016/j.jhevol.2015.06.013}\\[4pt]

2014 & {\bfseries Barr WA}. Functional Morphology of the Bovid Astragalus In Relation To Habitat: Controlling Phylogenetic Signal In Ecomorphology. \emph{Journal
of Morphology}. 275:1201-1216. \href{http://dx.doi.org/10.1002/jmor.20279}{doi:10.1002/jmor.20279}\\[4pt]

2014 & {\bfseries Barr WA} and Scott RS. Phylogenetic comparative methods complement discriminant function analysis in ecomorphology. \emph{American Journal
of Physical Anthropology}. 153:663-674. \href{http://dx.doi.org/10.1002/ajpa.22462}{doi:10.1002/ajpa.22462}\\[4pt]

2014 & Scott RS and {\bfseries Barr WA}. Ecomorphology and phylogenetic risk: implications for habitat reconstruction using fossil bovids.
\emph{Journal of Human Evolution}. 73:47-57. \href{http://dx.doi.org/10.1016/j.jhevol.2014.02.023}{doi:10.1016/j.jhevol.2014.02.023}\\[4pt]

2010 & Reed DN, and {\bfseries Barr WA}. A preliminary account of the rodents from Pleistocene levels at Grotte des Contrebandiers (Smuggler's Cave),
Morocco. \emph{Historical Biology}. 22:286-294. \href{http://dx.doi.org/10.1080/08912960903562192}{doi:10.1080/08912960903562192}\\
\end{longtable}

%exclude in prep
\begin{comment}
\begin{mylist}
\item[] {\bfseries In Prep}

\item[] {\bfseries Barr WA}.  Beyond 'Mosaic': mammalian functional diversity and habitat heterogeneity in human evolution. To be submitted to \emph{Journal of Human Evolution}.
\item[] {\bfseries Barr WA}.  The morphology of the bovid calcaneus: function, phylogeny, and allometric scaling. To be submitted to \emph{Journal of Morphology}.

\end{mylist}
\end{comment}


\section*{Honors and Awards}

\begin{tabular}{p{.14\textwidth}p{0.82\textwidth}}
2015 & Travel Grant - Paleoanthropology Society for meetings in Calgary. \$500.\\[4pt]
2013 & Pollitzer Student Travel Award - American Association of Physical Anthropologists. \$500.\\[4pt]
2008 - 2011 & Professional Development Award - Department of Anthropology, University of Texas at Austin.\\[4pt]
2007 & Student Prize - Texas Association of Biological Anthropologists.\\
\end{tabular}

\section*{Fieldwork }
\begin{longtable}{p{.16\textwidth}p{0.80\textwidth}}
2014 - Present & Mille-Logya Research Project, Afar Region, Ethiopia (Plio-Pleistocene). I conduct field research to recover new fossils and to understand the environmental and ecological context of human evolution in this region. \\[4pt]

2016 & Koobi-Fora Field School, East Turkana, Kenya. I collected fossil data relating to sub-regional faunal variability in the Koobi Fora Formation from 2.0 - 1.4 Ma. I supervised four undergraduate student research projects that were organized around this topic. \\[4pt]


2013 - 2014 & Great Divide Basin Project, Wyoming. Collected primate and mammalian fossils from Eocene sediments, and prospected for new localities. \\[4pt]

2010, 2012 & Dikika Research Project, Afar Region, Ethiopia. Surface collection of Plio- Pleistocene hominin and mammalian fossils. Managed GIS data collection with hand-held computers and high-precision GPS base station.\\[4pt]

2007, 2008, 2010 & Dalquest Research Site, Big Bend Region, Texas. Surface collected primate and mammalian fossils in the Devil's Graveyard Formation. (Eocene: Late Uintan).\\[4pt]

2009 & Contrebandiers Cave, Temara, Morocco.  Excavated site preserving Middle Stone Age archaeology (Aterian) and hominin remains. Performed systematic analysis of rodent fauna.\\
\end{longtable}

\section*{Synergistic Activities}
\begin{tabular}{p{.16\textwidth}p{0.8\textwidth}}

2015 - Present & External Member. Evolution of Terrestrial Ecosystems Working Group. National Museum of Natural History.\\[4pt]

2012 - Present & Research Associate and Software Developer. PaleoCore Project. I am a key member of this NSF Funded project, which aims to create a data-standard for physical anthropology. I contributed heavily to the development of PaleoCore informatics tools for data sharing.\\
\end{tabular}

\section*{Scholarly Presentations}
\subsection*{Invited Talks, Symposia, Workshops}

\begin{tabular}{p{.1\textwidth}p{0.86\textwidth}}
2017 & Invited workshop instructor. Topic: \emph{Data Analysis, Visualization, and Comparative Methods in R}. February 16-17. University of North Carolina - Greensboro.\\[4pt]

2015 & Invited participant in symposium entitled: \emph{Latest methods in reconstructing Cenozoic terrestrial environments and ecological communities}. September 10 - 12. Cleveland Museum of Natural History.\\[4pt]
2014 & Invited participant in symposium entitled: \emph{The Role of Mosaic Habitats in Hominin Evolution}. Annual Meeting of the American Association of Physical Anthropologists. Calgary, Alberta.\\
\end{tabular}
\subsection*{Published Abstracts from Conference Presentations}

\begin{mylist}
\item[] *indicates undergraduate under my supervision
\end{mylist}

\begin{longtable}{p{.1\textwidth}p{0.86\textwidth}}
2017 & {\bfseries Barr WA}. Bovid locomotor traits track land cover and mean annual precipitation: using an ecometric approach to reconstruct paleoenvironments in the Shungura Formation (Plio-Pleistocene, Ethiopia). American Association of Physical Anthropology.\\[4pt]
2017 & Llera C*, Benitez L*, Biernat M*, Braun DR,  Hammond AS, Patterson DB, and {\bfseries Barr WA}. Subregion-scale heterogeneity in bovid abundance in the Koobi Fora Formation (Pleistocene, Northern Kenya).  American Association of Physical Anthropology.\\[4pt]
2017 & Thompson B, Arenson J, Biernat M*,  {\bfseries Barr WA}, Reeves J, Braun DR and Hammond AS. A preliminary study of primate abundance in East Turkana collection areas relative to outcrop size. American Association of Physical Anthropology.\\[4pt]
2017 & Enny A*, Biernat M*, Braun DR, Reda W*, Hammond AS, Patterson DB and {\bfseries Barr WA}. Exploring the impact of collection strategies on interpretations of faunal abundance: a case study from the Koobi Fora Formation (Pleistocene, northern Kenya). American Association of Physical Anthropology.\\[4pt]
2017 & Benitez L*, Llera* C, Biernat M*, Braun DR, Hammond AS, Patterson DB, {\bfseries Barr WA}. The Implications of Faunal Abundance for Pleistocene Paleoenvironments in the Turkana Basin, Northern Kenya. Paleoanthropology Society. \\[4pt]

2016 & {\bfseries Barr WA}. Signal or noise? Testing hypotheses about faunal turnover. Paleoanthropology Society. \\[4pt]

2015 & {\bfseries Barr WA} and Dunn RH. A method for analyzing complex joint surfaces in ecomorphology using slope rasters derived from Digital Elevation Models. American Association of Physical Anthropology.\\[4pt]
2015 & Thompson JC, McPherron SP, Bobe R, {\bfseries Barr WA}, Reed D, Wynn J, Marean CW, and Alemseged Z. Taphonomy of fossils from the hominin-bearing deposits at Dikika, Ethiopia. Paleoanthropology Society.\\[4pt]

2014 & {\bfseries Barr WA}. Paleoenvironments of the Hadar and Shungura Formations: Synthesizing multiple lines of evidence using bovid ecomorphology. American Association of Physical Anthropology.\\[4pt]
2014 & Kemp A and {\bfseries Barr WA}. Rates of homoplasy in the mammalian skeleton. American Association of Physical Anthropology.\\[4pt]

2013 & {\bfseries Barr WA}. Ecomorphology of the bovid astragalus: body size, function, phylogeny and paleoenvironmental reconstruction. \emph{American Journal of Physical Anthropology}. 150:74.\\[4pt]

2012 & {\bfseries Barr WA}. Ecomorphology in a phylogenetic statistical context: a case study using the bovid femur. \emph{American Journal of Physical Anthropology}. 147:90-91.\\[4pt]

2012 & Scott RS and {\bfseries Barr WA}. Ecomorphology and phylogeny among the Bovidae: implications for habitat reconstruction. \emph{American Journal of Physical Anthropology}.\\[4pt]

2012 & Kappelman JK, Keane P, Reed D, Tenbarge J, Witzel A, {\bfseries Barr WA}, Nachman BA, Russo GA. eFossils.org: a collaborative website and community database for the study of human evolution. \emph{American Journal of Physical Anthropology}.\\[4pt]

2011 & Reed DN, McPherron S, {\bfseries Barr WA}, Alemseged Z, Bobe R, Geraads D, and Wynn J. A new GPS data collection methodology and data schema for integrating multiple project databases: examples from the Dikika Research Project geodatabase. \emph{American Journal of Physical Anthropology}. 144:249-250.\\[4pt]

2009 & {\bfseries Barr WA}, Reed DN. Coping with taxonomic ambiguity and inter-observer variation in paleontological and paleoanthropological analyses. \emph{American Journal of Physical Anthropology}. 144:249-250.\\[4pt]

2009 & Toborowsky CJ, {\bfseries Barr WA}, Lewis, RJ. Does environmental unpredictability drive lemur life histories? \emph{American Journal of Physical Anthropology}.\\[4pt]

2008 & {\bfseries Barr WA}. The effects of allometric scaling patterns on the template method for estimating dimorphism. \emph{American Journal of Physical Anthropology}.\\
\end{longtable}

\subsection*{Scholarly Presentations Without Published Abstracts}


\begin{longtable}{p{.1\textwidth}p{0.86\textwidth}}
2013 & {\bfseries Barr WA}, Nachman B, Shapiro L. The Academic Phylogeny of Physical Anthropology. Annual Meetings of the Texas Association for Biological Anthropologists. Austin, TX.\\[4pt]

2013 & Reed D, {\bfseries Barr WA}, Urban T. Free as in speech and beer: open source software solutions for spatial data management in physical anthropology. Annual Meetings of the Texas Association for Biological Anthropologists. Austin, TX.\\[4pt]

2012 & {\bfseries Barr WA}. Refining hominin paleoenvironmental reconstructions using bovid ecomorphology: the role of phylogenetic comparative methods. Presentation at University of Texas at Austin Paleontology Brown-Bag Seminar Series.\\[4pt]

2012 & {\bfseries Barr WA}. Refining hominin paleoenvironmental reconstructions with bovid ecomorphology. Presentation at University of Texas at Austin Informal Physical Anthropology Semininar series.\\[4pt]

2011 & {\bfseries Barr WA}. Ecomorphology in a phylogenetic statistical context: a case study using the bovid femur. Annual Meetings of Texas Association for Biological Anthropologists. San Marcos, TX.\\[4pt]

2010 & {\bfseries Barr WA}. Pattern or chaos? Exploring a null model of faunal turnover patterns. Annual Meetings of Texas Association for Biological Anthropologists. Waco, TX.\\[4pt]

2010 & {\bfseries Barr WA}. Quantitative ecomorphology of mammalian dentitions: Refining a tool for reconstructing early hominin paleoenvironments. Presentation at University of Texas at Austin Informal Physical Anthropology Semininar series.\\[4pt]

2008 & {\bfseries Barr WA}. Coping with taxonomic ambiguity and inter-observer variation in paleontological and paleoanthropological analyses. Annual Meetings of Texas Association for Biological Anthropologists. College Station, TX.\\[4pt]

2007 & {\bfseries Barr WA}. The effects of allometric scaling patterns on the template method for estimating dimorphism. Annual Meetings of Texas Association for Biological Anthropologists. Austin, TX.\\
\end{longtable}



\section*{Courses Taught}
\begin{mylist}

\item[]\emph{Introduction to Biological Anthropology}. ANTH 1001. Undergraduate survey of the field of biological anthropology. The George Washington University, Anthropology. Taught Fall 2016, Spring 2017.

\item[] \emph{Analytical Methods in Evolutionary Anthropology}. ANTH 6413. I designed this graduate course course covering applied statistical methods (e.g, regression, ANOVA and related techniques, categorical data analysis, resampling approaches) and the R statistical programming language. This is a required course for the Hominid Paleobiology PhD program. The George Washington University, Anthropology. Taught Spring 2015, Fall 2016.

\item[] \emph{Climate Change and Human Evolution}. ANTH 3491. I designed this upper level undergraduate course covering changes in global climate through evolutionary time and the impacts on evolution, with an emphasis on humans. The George Washington University, Anthropology. Taught Spring 2016, Spring 2017.

\item[] \emph{Public Understanding of Science}. HOMP 8302.  Graduate course in which students complete semester-long public service internships. Student projects target underserved Washington, DC-area public schools and general audiences at public museums with a goal of increasing scientific literacy and creating interest in scientific careers. The George Washington University, Anthropology. Taught Spring 2016.

\item[] \emph{GIS and Remote Sensing for Archaeology and Paleontology}. ANT 391 / GRG 396. Teaching Assistant.  University of Texas at Austin, Anthropology. 2010.

\item[] \emph{Human Variation}. ANT 394C. Teaching Assistant. University of Texas at Austin, Anthropology. 2009.

\item[] \emph{Introduction to Physical Anthropology}. ANT 301. Teaching Assistant. University of Texas at Austin, Anthropology. 2006, 2007, 2010, 2011, 2012.
\end{mylist}

\section*{Student Advising}
\subsection*{PhD students, committee member}
\begin{tabular}{p{.14\textwidth}p{0.82\textwidth}}
2017 & Chrisandra Kufeldt (The George Washington University)\\[4pt]
2016 & David Patterson (The George Washington University)\\[4pt]
In progress & Laurence Dumouchel (The George Washington University)\\[4pt]
In progress & Vance Powell (The George Washington University)\\
\end{tabular}

\subsection*{Undergraduate Assistants}
\begin{tabular}{p{.14\textwidth}p{0.82\textwidth}}
2016 - 2017 & Maryse Biernat (Stockton University)\\[4pt]
2016 - 2017 & Elliot Greiner (The George Washington University)\\
\end{tabular}

\subsection*{Koobi Fora Field School Students}
\begin{tabular}{p{.14\textwidth}p{0.82\textwidth}}
2016 & Lorena Benitez (Harvard University)\\[4pt]
2016 & Alyssa Enny (Stockton University)\\[4pt]
2016 & Catherine Llera (University of Florida)\\[4pt]
2016 & Weldeyared Reda (Aksum University, Ethiopia)\\
\end{tabular}

\pagebreak

\section*{Professional Service}
\begin{longtable}{p{.14\textwidth}p{0.82\textwidth}}
2017 & Webmaster, Center for the Advanced Study of Human Paleobiology. I maintain the website and mailing list for our research center.\\[4pt]
2014 - 2015 & Coordinator, Capstone Seminar - Coordinated speakers and organize the academic program for this departmental seminar in the Center for the Advanced Study of Human Paleobiology at The George Washington University. \\[4pt]
2013 - present &  \href{https://www.physanthphylogeny.org}{Academic Phylogeny of Physical Anthropology} - In collaboration with Liza Shapiro and Brett Nachman, I created this website as a public resource that tracks academic lineages of Physical Anthropology PhDs. The site has had over 2200 user submissions.\\[4pt]

 2008 & Reviewer - University of Texas Liberal Arts Graduate Research Fellowship. Evaluated grant proposals from students competing for \$50,000 in grant funds.\\
\end{longtable}

\section*{Public Outreach and Science Communication}
\begin{longtable}{p{.14\textwidth}p{0.82\textwidth}}
2017 & \emph{Survivors: What Fossils Tell Us About the Past and Future}. I answered questions for the general public at the National Museum of National History as part of the 30th anniversary celebration of the Evolution of Terrestrial Ecosystems program, of which I am a member. \\[4pt]
2016 - Present & \href{http://facesoffieldwork.com}{Faces of Fieldwork} - I launched this website to show the general public a more personal face of scientific fieldwork. Features weekly photographic posts highlighting the good, the bad, and the ugly of real people doing real research.\\[4pt]
2014-2015 & Volunteer, Explore UT - University wide K-12 educational open house. Helped organize and run activity ``Leaping Lemurs of Madagascar'' on locomotion and conservation of lemurs.\\
\end{longtable}

\section*{Media Coverage}
\begin{longtable}{p{.14\textwidth}p{0.82\textwidth}}
2017 &  Online coverage of \emph{Signal or noise? A null model method for testing hypotheses about pulsed faunal turnover} in \href{https://www.sciencedaily.com/releases/2017/08/170804100410.htm}{Science Daily}, \href{https://phys.org/news/2017-08-paper-genus-homo-response-environmental.html}{phys.org}, \href{https://gwtoday.gwu.edu/origin-human-genus-may-have-occurred-chance}{GW Today} and others. \\[4pt]
2017 & Feature on \href{https://www.theguardian.com/lifeandstyle/2017/jul/01/pregnant-in-the-field-blog-photography-have-trowel-will-travel}{theguardian.com} highlighting several contributors to Faces of Fieldwork.\\[4pt]
2015 & \emph{GWU aims to be among top research schools.} \href{http://www.washingtonpost.com/local/education/gwu-aims-to-be-among-top-research-schools/2015/03/03/491da24e-c1f1-11e4-9ec2-b418f57a4a99_gallery.html}{Washington Post}.\\
\end{longtable}

\section*{Manuscript Reviews}
\begin{tabular}{l l}
2014, 2015, 2017 & Journal of Human Evolution\\[4pt]
2017 & Comptes Rendus Palevol\\[4pt]
2016 & Nature Ecology \& Evolution\\[4pt]
2014 & International Journal of Primatology\\[4pt]
2012 & Methods in Ecology and Evolution\\[4pt]
2012 & Manning Publications (book proposal review)\\
\end{tabular}

\section*{Professional Memberships}
\begin{mylist}
\item[] American Association of Physical Anthropologists
\item[] Paleoanthropology Society
\end{mylist}

%\section*{References}
%{\bfseries Robert Anemone, PhD}\\
%Professor and Department Head\\
%Department of Anthropology\\
%The University of North Carolina - Greensboro\\
%426 Graham Building\\
%PO Box 26170\\
%Greensboro, NC 27412-5000\\
%Phone: (336) 334-5132\\
%Email: robert.anemone@uncg.edu\\
%\\
%{\bfseries Liza Shapiro, PhD}\\
%Professor\\
%Department of Anthropology\\
%University of Texas at Austin\\
%2201 Speedway Stop C3200\\
%Austin, TX 78712\\
%Phone: (512) 471-7533\\
%Email: liza.shapiro@mail.utexas.edu\\
%\\
%{\bfseries Chet Sherwood, PhD}\\
%Professor and Chair\\
%Department of Anthropology\\
%The George Washington University\\
%Science and Engineering Hall \\
%800 22nd Street NW Suite 6000\\
%Washington DC 20052\\
%Phone: (202) 994-6346\\
%Email: sherwood@email.gwu.edu\\

\end{document}
