\documentclass{article}
\usepackage[top=0.85in, bottom=0.9in, left=0.9in, right=0.9in]{geometry}
\usepackage{mdwlist}
\usepackage{fancyhdr}
\usepackage[hidelinks]{hyperref}
\usepackage{helvet}
\renewcommand{\familydefault}{\sfdefault}
\pagestyle{fancy}
\pagenumbering{arabic}
\lhead{\itshape W. Andrew Barr - CV}
\chead{}
\rhead{\thepage}
\lfoot{}
\cfoot{Updated \today}
\rfoot{}

\thispagestyle{empty}

\begin{document}
\begin{center}
\noindent{\bfseries{\Huge W. Andrew Barr - Curriculum Vitae}}
\end{center}

\vspace{15pt}

\noindent\begin{minipage}{.60\textwidth}
\begin{flushleft}
Center for the Advanced Study of Human Paleobiology\\
Department of Anthropology\\
The George Washington University\\
\end{flushleft}
\end{minipage}
\begin{minipage}{.395\textwidth}
\begin{flushright}
800 22nd St NW, Suite 6000\\
Washington, DC 20052 \\
wabarr@gmail.com - (202) 994-3213\\
\end{flushright}
\end{minipage}


\noindent\rule[-2mm]{\textwidth}{1pt}

\section*{Education}
\begin{description*}
\item[] PhD - 2014 - University of Texas at Austin. Anthropology.
\item[] MA  - 2008 - University of Texas at Austin. Anthropology.
\item[] BS  - 2005 - Tulane University. Anthropology, French.
\end{description*}


\section*{Academic Appointments}
\begin{description*}
\item[] {\bfseries Visiting Assistant Professor}. Center for the Advanced Study of Human Paleobiology. Department of Anthropology. The George Washington University. Spring 2016 - Present.
\item[] {\bfseries Postdoctoral Scientist}. Center for the Advanced Study of Human Paleobiology. Department of Anthropology. The George Washington University. Advisor: Bernard Wood. Fall 2014 - Fall 2015.
\item[] {\bfseries Research Associate}. Department of Paleobiology.  National Museum of Natural History. 2014 - Present.
\end{description*}


\section*{Peer Reviewed Publications}

\begin{description*}
\item[] {\bfseries In Review}
%\item[] Schilder B, {\bfseries Barr WA}, Bobe R, Sherwood C. Global climate influenced the evolutionary history of brain size increase in some mammalian lineages, but not in hominins. In review in \emph{Evolution}.
\item[] {\bfseries Barr WA}. Signal or noise? A null model method for testing hypotheses about pulsed faunal turnover. In review in \emph{Methods in Ecology and Evolution.} 
\item[] Kemp A and {\bfseries Barr WA}. Rates of homoplasy vary among regions of the mammalian skeleton. In review in \emph{Journal of Human Evolution.} 
\end{description*}

\begin{description*}
\item[] {\bfseries Accepted or In Press}
\item[] {\bfseries Barr WA}. In Press. Bovid locomotor functional trait distributions reflect land cover and annual precipitation in sub-Saharan Africa. \emph{Evolutionary Ecology Research}.
\item[] {\bfseries Barr WA}. Accepted.  \emph{Ecomorphology}. To be published in D.A. Croft, S.W. Simpson, and D.F. Su (eds.),  \emph{Methods in Paleoecology: Reconstructing Cenozoic Terrestrial Environments and Ecological Communities}. Springer (Vertebrate Paleobiology and Paleoanthropology Series), Dordrecht.
\end{description*}

\begin{description*}
\item[] {\bfseries 2015}
\item[] {\bfseries Barr WA}. Paleoenvironments of the Shungura Formation (Plio-Pleistocene: Ethiopia) based on ecomorphology of the bovid astragalus. \emph{Journal of Human Evolution}. 88:97-107. \href{http://dx.doi.org/10.1016/j.jhevol.2015.05.002}{doi:10.1016/j.jhevol.2015.05.002}
\item Reed D, {\bfseries Barr WA}, McPherron S, Bobe R, Geraads D, Wynn J, Alemseged Z. Digital Data Collection in Paleoanthropology. \emph{Evolutionary Anthropology}. 24:238-249. \href{http://dx.doi.org/10.1002/evan.21466}{doi:10.1002/evan.21466}
\item[] Thompson JC, McPherron S, Bobe R, Reed DN, {\bfseries Barr WA}, Wynn J, Marean CW, Geraads D, Alemseged Z. Taphonomy of fossils from the hominin-bearing deposits at Dikika, Ethiopia. \emph{Journal of Human Evolution}. 86:112-135. \href{http://dx.doi.org/10.1016/j.jhevol.2015.06.013}{doi:10.1016/j.jhevol.2015.06.013}


\end{description*}

\begin{description*}
\item[] {\bfseries 2014}
\item[] {\bfseries Barr WA}. Functional Morphology of the Bovid Astragalus In Relation To Habitat: Controlling Phylogenetic Signal In Ecomorphology. \emph{Journal
of Morphology}. 275:1201-1216. \href{http://dx.doi.org/10.1002/jmor.20279}{doi:10.1002/jmor.20279}

\item[] {\bfseries Barr WA} and Scott RS. Phylogenetic comparative methods complement discriminant function analysis in ecomorphology. \emph{American Journal
of Physical Anthropology}. 153:663-674. \href{http://dx.doi.org/10.1002/ajpa.22462}{doi:10.1002/ajpa.22462}

\item[] Scott RS and {\bfseries Barr WA}. Ecomorphology and phylogenetic risk: implications for habitat reconstruction using fossil bovids.
\emph{Journal of Human Evolution}. 73:47-57. \href{http://dx.doi.org/10.1016/j.jhevol.2014.02.023}{doi:10.1016/j.jhevol.2014.02.023}

\end{description*}

\begin{description*}
\item[] {\bfseries 2010}
\item[] Reed DN, and {\bfseries Barr WA}. A preliminary account of the rodents from Pleistocene levels at Grotte des Contrebandiers (Smuggler's Cave),
Morocco. \emph{Historical Biology}. 22:286-294. \href{http://dx.doi.org/10.1080/08912960903562192}{doi:10.1080/08912960903562192}
\end{description*}


\section*{Funding and Awards}

%\begin{minipage}{\linewidth}
%\begin{description*}
%\item[] {\bfseries Pending}
%\item[] National Science Foundation (RIDIR) - \emph{AnthroCore}: An open-source spatial database infrastructure for biological anthropology, archaeology and other field and collection based research in the social and behavioral sciences. PIs: Reed, DiFiore, Kappelman. Role: Senior Scientist. Amount requested: \$1,588,069.
%\end{description*}
%\end{minipage}


\begin{description*}
\item[] {\bfseries 2015}
\item[] National Science Foundation (Archaeology) - Middle Pleistocene Hominin Behavior and Paleoecology at Farre, Chalbi Basin, Northern Kenya. PI: Ferraro. Role: Senior Scientist.  Amount awarded: \$60,000.
\end{description*}


\begin{description*}
\item[] {\bfseries 2014}
\item[] Travel Grant - Paleoanthropology Society for meetings in Calgary. \$500.
\end{description*}


\begin{description*}
\item[] {\bfseries 2013}
\item[] Named Continuing Fellowship - University of Texas at Austin Graduate School. \$29,000.
\item[] Pollitzer Student Travel Award - American Association of Physical Anthropologists. \$500.
\end{description*}


\begin{description*}
\item[] {\bfseries 2012}
\item[] Dissertation Fieldwork Grant - Wenner-Gren Foundation. \$13,317.
\end{description*}

\begin{description*}
\item[] {\bfseries 2008}
\item[] Professional Development Award - Department of Anthropology, University of Texas at Austin. Also recieved in 2009-2011.
\end{description*}

\begin{description*}
\item[] {\bfseries 2007}
\item[] Graduate Research Fellowship - National Science Foundation. \$90,000.

\item[] Liberal Arts Graduate Research Fellowship - University of Texas at Austin.

\item[] Student Prize - Texas Association of Biological Anthropologists.

\item[] David Bruton, Jr. Graduate Fellowship - University of Texas at Austin. Also received in 2008.
\end{description*}


\section*{Courses Taught}
\begin{description*}

\item[]\emph{Introduction to Biological Anthropology}. ANTH 1001. Undergraduate survey course with enrollment of approximately 240 students. This course is designed as a survey of the field of biological anthropology, and fulfills a general science requirement. George Washington University, Anthropology. Taught Fall 2016.

\item[] \emph{Analytical Methods in Evolutionary Anthropology}. ANTH 6413. I designed this graduate course course covering applied statistical methods (e.g, regression, ANOVA and related techniques, categorical data analysis, resampling approaches) and the R statistical programming language. This is a required course for the Hominid Paleobiology PhD program. George Washington University, Anthropology. Taught Spring 2015, Spring 2016.

\item[] \emph{Climate Change and Human Evolution}. ANTH 3491. I designed this upper level undergraduate course covering changes in global climate through evolutionary time and the impacts on evolution, with an emphasis on humans. (Previous course title: Evolutionary Impacts of Cenozoic Climate Change). George Washington University, Anthropology. Taught Spring 2016.

\item[] \emph{Public Understanding of Science}. HOMP 8302.  Graduate course in which students complete semester-long public service internships. Student projects target underserved Washington, DC-area public schools and general audiences at public museums with a goal of increasing scientific literacy and creating interest in scientific careers. George Washington University, Anthropology. Taught Spring 2016.

\item[] \emph{GIS and Remote Sensing for Archaeology and Paleontology}. ANT 391 / GRG 396. Teaching Assistant.  University of Texas at Austin, Anthropology. 2010.

\item[] \emph{Human Variation}. ANT 394C. Teaching Assistant. University of Texas at Austin, Anthropology. 2009.

\item[] \emph{Introduction to Physical Anthropology}. ANT 301. Teaching Assistant. University of Texas at Austin, Anthropology. 2006, 2007, 2010, 2011, 2012.
\end{description*}

\section*{Professional Preparation}
\begin{description*}

\item[] External Member. Evolution of Terrestrial Ecosystems Working Group. National Museum of Natural History. 2015 - Present.

\item[] Research Associate and Software Developer. PaleoCore Project. I am a key member of this NSF Funded project, which aims to create a data-standard for physical anthropology. I contributed heavily to the development of PaleoCore informatics tools for data sharing. 2012 - Present.

\item[] Participant. AnthroTree Workshop in Phylogenetic Methods. Amherst, MA. 2011.

\item[] Research Assistant. eSkeletons.org and eFossils.org. PI: John Kappelman. 2010, 2011.

\end{description*}


\section*{Fieldwork Experience}
\begin{description*}

\item[] Koobi-Fora Field School, East Turkana, Kenya. Directors: David Braun and Purity Kiura. I collected fossil data relating to sub-regional faunal variability in the Koobi Fora Formation from 2.0 - 1.4 Ma. I supervised four undergraduate student research projects that were organized around this topic. Summer 2016. 

\item[] Mille-Logya Research Project, Afar Region, Ethiopia. PI: Zeresenay Alemseged. I am a member of the scientific team conducting research to recover new fossil evidence of human evolution from this Plio-Pleistocene site. 2014.

\item[] Great Divide Basin Project, Wyoming. PI: Robert Anemone. Collected primate and mammalian fossils from Eocene sediments, and prospected for new localities. 2013, 2014.

\item[] Dikika Research Project, Afar Region, Ethiopia. PI: Zeresenay Alemseged. Surface collection of Plio- Pleistocene hominin and mammalian fossils. Managed GIS data collection with hand-held computers and high-precision GPS base station. 2010, 2012.

\item[] Dalquest Research Site, Big Bend Region, Texas. PI: E. Chris Kirk. Surface collected primate and mammalian fossils in the Devil's Graveyard Formation. (Eocene: Late Uintan). 2007, 2008, 2010.

\item[] Contrebandiers Cave, Temara, Morocco. PI: Harrold Dibble \& Utsav Schurmans. Excavated site preserving Middle Stone Age archaeology (Aterian) and hominin remains. Performed systematic analysis of rodent fauna. 2009.
\end{description*}

\section*{Scholarly Presentations}

\subsection*{Published Abstracts from Conference Presentations}

\begin{description*}
\begin{minipage}{\linewidth}
\item[] {\bfseries 2016}
\item[] {\bfseries Barr WA}. Signal or noise? Testing hypotheses about faunal turnover. Paleoanthropology Society. 
\end{minipage}
\end{description*}



\begin{description*}
\begin{minipage}{\linewidth}
\item[] {\bfseries 2015}
\item[] {\bfseries Barr WA} and Dunn RH. A method for analyzing complex joint surfaces in ecomorphology using slope rasters derived from Digital Elevation Models. American Association of Physical Anthropology.
\item[] Thompson JC, McPherron SP, Bobe R, {\bfseries Barr WA}, Reed D, Wynn J, Marean CW, and Alemseged Z. Taphonomy of fossils from the hominin-bearing deposits at Dikika, Ethiopia. Paleoanthropology Society.
\end{minipage}
\end{description*}



\begin{description*}
\begin{minipage}{\linewidth}
\item[] {\bfseries 2014}
\item[] {\bfseries Barr WA}. Paleoenvironments of the Hadar and Shungura Formations: Synthesizing multiple lines of evidence using bovid ecomorphology. American Association of Physical Anthropology.
\item[] Kemp A and {\bfseries Barr WA}. Rates of homoplasy in the mammalian skeleton. American Association of Physical Anthropology.
\end{minipage}
\end{description*}



\begin{description*}
\begin{minipage}{\linewidth}
\item[] {\bfseries 2013}
\item[] {\bfseries Barr WA}. Ecomorphology of the bovid astragalus: body size, function, phylogeny and paleoenvironmental reconstruction. \emph{American Journal of Physical Anthropology}. 150:74.
\end{minipage}
\end{description*}



\begin{description*}
\begin{minipage}{\linewidth}
\item[] {\bfseries 2012}
\item[] {\bfseries Barr WA}. Ecomorphology in a phylogenetic statistical context: a case study using the bovid femur. \emph{American Journal of Physical Anthropology}. 147:90-91.

\item[] Scott RS and {\bfseries Barr WA}. Ecomorphology and phylogeny among the Bovidae: implications for habitat reconstruction. \emph{American Journal of Physical Anthropology}.

\item[] Kappelman JK, Keane P, Reed D, Tenbarge J, Witzel A, {\bfseries Barr WA}, Nachman BA, Russo GA. eFossils.org: a collaborative website and community database for the study of human evolution. \emph{American Journal of Physical Anthropology}.
\end{minipage}
\end{description*}



\begin{description*}
\begin{minipage}{\linewidth}
\item[] {\bfseries 2011}
\item[] Reed DN, McPherron S, {\bfseries Barr WA}, Alemseged Z, Bobe R, Geraads D, and Wynn J. A new GPS data collection methodology and data schema for integrating multiple project databases: examples from the Dikika Research Project geodatabase. \emph{American Journal of Physical Anthropology}. 144:249-250.
\end{minipage}
\end{description*}



\begin{description*}
\begin{minipage}{\linewidth}
\item[] {\bfseries 2009}
\item[] {\bfseries Barr WA}, Reed DN. Coping with taxonomic ambiguity and inter-observer variation in paleontological and paleoanthropological analyses. \emph{American Journal of Physical Anthropology}. 144:249-250.

\item[] Toborowsky CJ, {\bfseries Barr WA}, Lewis, RJ. Does environmental unpredictability drive lemur life histories? \emph{American Journal of Physical Anthropology}.
\end{minipage}
\end{description*}



\begin{description*}
\begin{minipage}{\linewidth}
\item[] {\bfseries 2008}
\item[] {\bfseries Barr WA}. The effects of allometric scaling patterns on the template method for estimating dimorphism. \emph{American Journal of Physical Anthropology}.
\end{minipage}
\end{description*}

\subsection*{Scholarly Presentations Without Published Abstracts}


\begin{description*}
\begin{minipage}{\linewidth}
\item[] {\bfseries 2013}
\item[] {\bfseries Barr WA}, Nachman B, Shapiro L. The Academic Phylogeny of Physical Anthropology. Annual Meetings of the Texas Association for Biological Anthropologists. Austin, TX.

\item[] Reed D, {\bfseries Barr WA}, Urban T. Free as in speech and beer: open source software solutions for spatial data management in physical anthropology. Annual Meetings of the Texas Association for Biological Anthropologists. Austin, TX.
\end{minipage}
\end{description*}



\begin{description*}
\begin{minipage}{\linewidth}
\item[] {\bfseries 2012}
\item[] {\bfseries Barr WA}. Refining hominin paleoenvironmental reconstructions using bovid ecomorphology: the role of phylogenetic comparative methods. Presentation at University of Texas at Austin Paleontology Brown-Bag Seminar Series.

\item[] {\bfseries Barr WA}. Refining hominin paleoenvironmental reconstructions with bovid ecomorphology. Presentation at University of Texas at Austin Informal Physical Anthropology Semininar series.
\end{minipage}
\end{description*}


\begin{description*}
\begin{minipage}{\linewidth}
\item[] {\bfseries 2011}
\item[] {\bfseries Barr WA}. Ecomorphology in a phylogenetic statistical context: a case study using the bovid femur. Annual Meetings of Texas Association for Biological Anthropologists. San Marcos, TX.
\end{minipage}
\end{description*}


\begin{description*}
\item[] {\bfseries 2010}
\item[] {\bfseries Barr WA}. Pattern or chaos? Exploring a null model of faunal turnover patterns. Annual Meetings of Texas Association for Biological Anthropologists. Waco, TX.

\item[] {\bfseries Barr WA}. Quantitative ecomorphology of mammalian dentitions: Refining a tool for reconstructing early hominin paleoenvironments. Presentation at University of Texas at Austin Informal Physical Anthropology Semininar series.
\end{description*}
\begin{description*}
\item[] {\bfseries 2008}
\item[] {\bfseries Barr WA}. Coping with taxonomic ambiguity and inter-observer variation in paleontological and paleoanthropological analyses. Annual Meetings of Texas Association for Biological Anthropologists. College Station, TX.
\end{description*}
\begin{description*}
\item[] {\bfseries 2007}
\item[] {\bfseries Barr WA}. The effects of allometric scaling patterns on the template method for estimating dimorphism. Annual Meetings of Texas Association for Biological Anthropologists. Austin, TX.
\end{description*}


\subsection*{Invited Talks, Symposia, Guest Lectures}

\begin{description*}
\item[] {\bfseries 2015}
\item[] Invited participant in symposium entitled: \emph{Latest methods in reconstructing Cenozoic terrestrial environments and ecological communities}. Cleveland Museum of Natural History. September 10 - 12.
\item[] {\bfseries 2013}
\item[]  Guest Lecture for Denn\'{e} Reed in graduate-level Statistical Methods course at University of Texas at Austin. Topic: Data reshaping and advanced plotting with ggplot2.
\item[] Guest Lecture for E. Christopher Kirk Introduction to Physical Anthropology course at University of Texas at Austin. Topic: Primate Evolution.
\end{description*}


\section*{Professional Service}
\begin{description*}
\item[] Coordinator, Capstone Seminar - I coordinate speakers and organize the academic program for this departmental seminar in the Center for the Advanced Study of Human Paleobiology at The George Washington University.

\item[] \href{http://www.physanthphylogeny.org}{Academic Phylogeny of Physical Anthropology} - In collaboration with Liza Shapiro and Brett Nachman, I created this website as a public resource that tracks academic lineages of Physical Anthropology PhDs. The site has had over 1700 user submissions. 2013.

\item[] Volunteer, Explore UT - University wide K-12 educational open house. Helped organize and run activity ``Leaping Lemurs of Madagascar'' on locomotion and conservation of lemurs. 2010, 2011.

\item[] Reviewer - University of Texas Liberal Arts Graduate Research Fellowship. Evaluated grant proposals from students competing for \$50,000 in grant funds. 2008.

\item[] Coordinator - Informal Physical Anthropology Seminar Series, University of Texas at Austin. Responsible for planning weekly seminars and recruiting speakers. 2008.
\end{description*}

\subsection*{Manuscript Reviews}
\begin{description*}
\item[] Nature Ecology \& Evolution. 2016.
\item[] Journal of Human Evolution. 2014, 2015.
\item[] International Journal of Primatology. 2014.
\item[] Methods in Ecology and Evolution. 2012.
\item[] Manning Publications (book proposal review). 2012.
\end{description*}
\subsection*{Professional Memberships}
\begin{description*}
\item[] American Association of Physical Anthropologists
\item[] Paleoanthropology Society
\end{description*}
\end{document}
