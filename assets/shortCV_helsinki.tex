\documentclass{article}
\usepackage[top=0.85in, bottom=0.6in, left=0.6in, right=0.6in]{geometry}
\usepackage{mdwlist}
\usepackage{enumitem}
\usepackage{etaremune}
\usepackage{longtable}
\usepackage{verbatim}
\usepackage{fancyhdr}
\usepackage[colorlinks=True, urlcolor=cyan]{hyperref}
\usepackage{helvet}
\renewcommand{\familydefault}{\sfdefault}
\pagestyle{fancy}
\pagenumbering{arabic}
\lhead{\itshape W. Andrew Barr - CV}
\chead{}
\rhead{\thepage}
\lfoot{}
\cfoot{Updated \today}
\rfoot{}

\thispagestyle{empty}

%custom environment to control hanging indent in description
\newenvironment{mylist}
{\begin{description}[style=unboxed,leftmargin=1.3cm]}
{\end{description}}


\begin{document}
\begin{center}
\noindent{\bfseries{\Huge W. Andrew Barr - Curriculum Vitae}}
\end{center}

\vspace{15pt}

\noindent\begin{minipage}{.60\textwidth}
\begin{flushleft}
Center for the Advanced Study of Human Paleobiology\\
Department of Anthropology\\
The George Washington University\\
\end{flushleft}
\end{minipage}
\begin{minipage}{.395\textwidth}
\begin{flushright}
800 22nd St NW, Suite 6000\\
Washington, DC 20052 \\
wabarr@gmail.com - (202) 994-3213\\
\end{flushright}
\end{minipage}


\noindent\rule[-2mm]{\textwidth}{1pt}

\section*{Education}

\begin{tabular}{p{.1\textwidth}p{0.86\textwidth}}
2014 & University of Texas at Austin. Ph.D., Anthropology. \\[4pt] %\emph{The Paleoenvironments of Early Hominins in the Omo Shungura Formation (Plio-Pleistocene, Ethiopia): Synthesizing Multiple Lines of Evidence Using Phylogenetic Ecomorphology}.
2008 & University of Texas at Austin. M.A., Anthropology. \\[4pt]
2005 & Tulane University. B.S., Anthropology and French.\\
\end{tabular}


\section*{Academic Appointments}
\begin{tabular}{p{.15\textwidth}p{0.8\textwidth}}
2016 - Present & Visiting Assistant Professor. Center for the Advanced Study of Human Paleobiology. Department of Anthropology. The George Washington University.\\[4pt]
2014 - 2016 & Postdoctoral Scientist. Center for the Advanced Study of Human Paleobiology. Department of Anthropology. The George Washington University. Advisor: Bernard Wood.\\[4pt]
2014 - Present & Research Associate. Department of Paleobiology.  National Museum of Natural History.\\
\end{tabular}

\section*{Research Grants and Fellowships}
\begin{tabular}{p{.1\textwidth}p{0.86\textwidth}}
In Prep & High-Risk Research in Biological Anthropology - In the shadow of Mt. Kenya: paleontology and paleoanthropology of Laikipia, Kenya. Target Submission Date - Dec. 3rd. Role PI. \$30,000.\\[4pt]
In Review &  National Science Foundation - Collaborative Research: REU Site: Past and Present Human-Environment Dynamics in the Turkana Basin, Kenya. Role: Senior Personnel. \$305,846\\[4pt]
2018 & American Association of Physical Anthropologists - Professional Development Award. Tumbili (late Miocene, Kenya): A new window into eastern African mammalian evolution at the dawn of the hominin lineage. \$7,500\\[4pt]
2013 &  University of Texas at Austin - Named Continuing Fellowship. \$29,000.\\[4pt]
2012  & Wenner-Gren Foundation - Dissertation Fieldwork Grant. \$13,317.\\[4pt]
2007 &  National Science Foundation - Graduate Research Fellowship. \$90,000.\\[4pt]
2007 &  University of Texas at Austin - Liberal Arts Graduate Research Fellowship. \$1,800.\\[4pt]
%2007, 2008 &  David Bruton, Jr. Graduate Fellowship - University of Texas at Austin. \$1,000.\\[4pt]

\end{tabular}

\section*{Courses Taught}
\begin{mylist}

\item[]\emph{Introduction to Biological Anthropology}. ANTH 1001. Undergraduate survey of the field of biological anthropology. The George Washington University, Anthropology. Taught Fall 2016, Spring 2017.

\item[] \emph{Analytical Methods in Evolutionary Anthropology}. ANTH 6413. I designed this graduate course course covering applied statistical methods (e.g, regression, ANOVA and related techniques, categorical data analysis, resampling approaches) and the R statistical programming language. This is a required course for the Hominid Paleobiology PhD program. The George Washington University, Anthropology. Taught Spring 2015, Fall 2016, Fall 2018.

\item[] \emph{Climate Change and Human Evolution}. ANTH 3491. I designed this upper level undergraduate course covering changes in global climate through evolutionary time and the impacts on evolution, with an emphasis on humans. The George Washington University, Anthropology. Taught Spring 2016, Spring 2017.

\item[] \emph{Public Understanding of Science}. HOMP 8302.  Graduate course in which students complete semester-long public service internships. Student projects target underserved Washington, DC-area public schools and general audiences at public museums with a goal of increasing scientific literacy and creating interest in scientific careers. The George Washington University, Anthropology. Taught Spring 2016.

\item[] \emph{GIS and Remote Sensing for Archaeology and Paleontology}. ANT 391 / GRG 396. Teaching Assistant.  University of Texas at Austin, Anthropology. 2010.

\item[] \emph{Human Variation}. ANT 394C. Teaching Assistant. University of Texas at Austin, Anthropology. 2009.

\item[] \emph{Introduction to Physical Anthropology}. ANT 301. Teaching Assistant. University of Texas at Austin, Anthropology. 2006, 2007, 2010, 2011, 2012.
\end{mylist}

\section*{Honors and Awards}

\begin{tabular}{p{.14\textwidth}p{0.82\textwidth}}
2015 & Travel Grant - Paleoanthropology Society for meetings in Calgary. \$500.\\[4pt]
2013 & Pollitzer Student Travel Award - American Association of Physical Anthropologists. \$500.\\[4pt]
2008 - 2011 & Professional Development Award - Department of Anthropology, University of Texas at Austin.\\[4pt]
2007 & Student Prize - Texas Association of Biological Anthropologists.\\
\end{tabular}




\section*{PhD Student Advising (committee member)}
\begin{tabular}{p{.14\textwidth}p{0.82\textwidth}}
In Progress & Eve Boyle (The George Washington University)\\[4pt]
2018 & Laurence Dumouchel (The George Washington University)\\[4pt]
2018 & Vance Powell (The George Washington University)\\[4pt]
2017 & Chrisandra Kufeldt (The George Washington University)\\[4pt]
2016 & David Patterson (The George Washington University)\\

\end{tabular}

\section*{Public Outreach and Science Communication}
\begin{longtable}{p{.14\textwidth}p{0.82\textwidth}}
May 16, 2018 & \emph{The Scientist is In}. National Museum of Natural History. I interacted directly with over a hundred visitors to the National Museum of National History's Hall of Human Origins and answered their questions about human evolution.\\[4pt]
March 30, 2017 & \emph{Survivors: What Fossils Tell Us About the Past and Future}. I answered questions for the general public at the National Museum of National History as part of the 30th anniversary celebration of the Evolution of Terrestrial Ecosystems program, of which I am a member. \\[4pt]
2016 - Present & \href{http://facesoffieldwork.com}{Faces of Fieldwork} - I created Faces of Fieldwork because I believe people engage more with science when they understand who we are and what we do. This site features weekly photographic posts highlighting the good, the bad, and the ugly of real people doing real research.\\[4pt]
2014-2015 & Volunteer, Explore UT - University wide K-12 educational open house. Helped organize and run activity ``Leaping Lemurs of Madagascar'' on locomotion and conservation of lemurs.\\
\end{longtable}



\end{document}
