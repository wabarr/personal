\documentclass{article}
\usepackage[top=0.85in, bottom=0.9in, left=0.9in, right=0.9in]{geometry}
\usepackage{mdwlist}
\usepackage{fancyhdr}
\usepackage[hidelinks]{hyperref}
\usepackage{helvet}
\renewcommand{\familydefault}{\sfdefault}
\pagestyle{fancy}
\pagenumbering{arabic}
\lhead{\itshape W. Andrew Barr - CV}
\chead{}
\rhead{\thepage}
\lfoot{}
\cfoot{Updated \today}
\rfoot{}

\thispagestyle{empty}

\begin{document}
\begin{center}
\noindent{\bfseries{\Huge W. Andrew Barr - Curriculum Vitae}}
\end{center}

\vspace{15pt}

\noindent\begin{minipage}{.60\textwidth}
\begin{flushleft}
Center for the Advanced Study of Human Paleobiology\\
Department of Anthropology\\
The George Washington University\\
\end{flushleft}
\end{minipage}
\begin{minipage}{.395\textwidth}
\begin{flushright}
800 22nd St NW, Suite 6000\\
Washington, DC 20052 \\
wabarr@gmail.com - (202) 994-3213\\
\end{flushright}
\end{minipage}


\noindent\rule[-2mm]{\textwidth}{1pt}

\section*{Education}
\begin{description*}
\item[] PhD - 2014 - University of Texas at Austin. Anthropology.
\item[] MA  - 2008 - University of Texas at Austin. Anthropology.
\item[] BS  - 2005 - Tulane University. Anthropology, French.
\end{description*}


\section*{Academic Appointments}
\begin{description*}
\item[] {\bfseries Visiting Assistant Professor}. Center for the Advanced Study of Human Paleobiology. Department of Anthropology. The George Washington University. Spring 2016 - Present.
\item[] {\bfseries Postdoctoral Scientist}. Center for the Advanced Study of Human Paleobiology. Department of Anthropology. The George Washington University. Advisor: Bernard Wood. Fall 2014 - Fall 2015.
\item[] {\bfseries Research Associate}. Department of Paleobiology.  National Museum of Natural History. 2014 - Present.
\end{description*}


\section*{Peer Reviewed Publications}

%\begin{description*}
%\item[] {\bfseries In Prep}
%\item[] {\bfseries Barr WA}. Signal or noise? Testing hypotheses about faunal turnover.
%\end{description*}

\begin{description*}
\item[] {\bfseries In Review}
%\item[] Schilder B, {\bfseries Barr WA}, Bobe R, Sherwood C. Global climate influenced the evolutionary history of brain size increase in some mammalian lineages, but not in hominins. In review in \emph{Evolution}.
\item[] Kemp A and {\bfseries Barr WA}. Rates of homoplasy vary among regions of the mammalian skeleton. In review in \emph{Journal of Human Evolution.} 
\end{description*}

\begin{description*}
\item[] {\bfseries Accepted or In Press}
\item[] {\bfseries Barr WA}. In Press. Bovid locomotor functional trait distributions reflect land cover and annual precipitation in sub-Saharan Africa. \emph{Evolutionary Ecology Research}.
\item[] {\bfseries Barr WA}. Accepted.  \emph{Ecomorphology}. To be published in D.A. Croft, S.W. Simpson, and D.F. Su (eds.),  \emph{Methods in Paleoecology: Reconstructing Cenozoic Terrestrial Environments and Ecological Communities}. Springer (Vertebrate Paleobiology and Paleoanthropology Series), Dordrecht.
\end{description*}

\begin{description*}
\item[] {\bfseries 2015}
\item[] {\bfseries Barr WA}. Paleoenvironments of the Shungura Formation (Plio-Pleistocene: Ethiopia) based on ecomorphology of the bovid astragalus. \emph{Journal of Human Evolution}. 88:97-107. \href{http://dx.doi.org/10.1016/j.jhevol.2015.05.002}{doi:10.1016/j.jhevol.2015.05.002}
\item Reed D, {\bfseries Barr WA}, McPherron S, Bobe R, Geraads D, Wynn J, Alemseged Z. Digital Data Collection in Paleoanthropology. \emph{Evolutionary Anthropology}. 24:238-249. \href{http://dx.doi.org/10.1002/evan.21466}{doi:10.1002/evan.21466}
\item[] Thompson JC, McPherron S, Bobe R, Reed DN, {\bfseries Barr WA}, Wynn J, Marean CW, Geraads D, Alemseged Z. Taphonomy of fossils from the hominin-bearing deposits at Dikika, Ethiopia. \emph{Journal of Human Evolution}. 86:112-135. \href{http://dx.doi.org/10.1016/j.jhevol.2015.06.013}{doi:10.1016/j.jhevol.2015.06.013}


\end{description*}

\begin{description*}
\item[] {\bfseries 2014}
\item[] {\bfseries Barr WA}. Functional Morphology of the Bovid Astragalus In Relation To Habitat: Controlling Phylogenetic Signal In Ecomorphology. \emph{Journal
of Morphology}. 275:1201-1216. \href{http://dx.doi.org/10.1002/jmor.20279}{doi:10.1002/jmor.20279}

\item[] {\bfseries Barr WA} and Scott RS. Phylogenetic comparative methods complement discriminant function analysis in ecomorphology. \emph{American Journal
of Physical Anthropology}. 153:663-674. \href{http://dx.doi.org/10.1002/ajpa.22462}{doi:10.1002/ajpa.22462}

\item[] Scott RS and {\bfseries Barr WA}. Ecomorphology and phylogenetic risk: implications for habitat reconstruction using fossil bovids.
\emph{Journal of Human Evolution}. 73:47-57. \href{http://dx.doi.org/10.1016/j.jhevol.2014.02.023}{doi:10.1016/j.jhevol.2014.02.023}

\end{description*}

\begin{description*}
\item[] {\bfseries 2010}
\item[] Reed DN, and {\bfseries Barr WA}. A preliminary account of the rodents from Pleistocene levels at Grotte des Contrebandiers (Smuggler's Cave),
Morocco. \emph{Historical Biology}. 22:286-294. \href{http://dx.doi.org/10.1080/08912960903562192}{doi:10.1080/08912960903562192}
\end{description*}


\section*{Funding and Awards}

%\begin{minipage}{\linewidth}
%\begin{description*}
%\item[] {\bfseries Pending}
%\item[] National Science Foundation (RIDIR) - \emph{AnthroCore}: An open-source spatial database infrastructure for biological anthropology, archaeology and other field and collection based research in the social and behavioral sciences. PIs: Reed, DiFiore, Kappelman. Role: Senior Scientist. Amount requested: \$1,588,069.
%\end{description*}
%\end{minipage}


\begin{description*}
\item[] {\bfseries 2015}
\item[] National Science Foundation (Archaeology) - Middle Pleistocene Hominin Behavior and Paleoecology at Farre, Chalbi Basin, Northern Kenya. PI: Ferraro. Role: Senior Scientist.  Amount awarded: \$60,000.
\end{description*}


\begin{description*}
\item[] {\bfseries 2014}
\item[] Travel Grant - Paleoanthropology Society for meetings in Calgary. \$500.
\end{description*}


\begin{description*}
\item[] {\bfseries 2013}
\item[] Named Continuing Fellowship - University of Texas at Austin Graduate School. \$29,000.
\item[] Pollitzer Student Travel Award - American Association of Physical Anthropologists. \$500.
\end{description*}


\begin{description*}
\item[] {\bfseries 2012}
\item[] Dissertation Fieldwork Grant - Wenner-Gren Foundation. \$13,317.
\end{description*}

\begin{description*}
\item[] {\bfseries 2008}
\item[] Professional Development Award - University of Texas at Austin. Also recieved in 2009-2011.
\end{description*}

\begin{description*}
\item[] {\bfseries 2007}
\item[] Graduate Research Fellowship - National Science Foundation. \$90,000.


\item[] David Bruton, Jr. Graduate Fellowship - University of Texas at Austin. Also received in 2008.
\end{description*}


\section*{Courses Taught}
\begin{description*}

\item[]\emph{Introduction to Biological Anthropology}. ANTH 1001. Undergraduate survey course with enrollment of approximately 240 students. This course is designed as a survey of the field of biological anthropology, and fulfills a general science requirement. George Washington University, Anthropology. Taught Fall 2016.

\item[] \emph{Analytical Methods in Evolutionary Anthropology}. ANTH 6413. I designed this graduate course course covering applied statistical methods (e.g, regression, ANOVA and related techniques, categorical data analysis, resampling approaches) and the R statistical programming language. This is a required course for the Hominid Paleobiology PhD program. George Washington University, Anthropology. Taught Spring 2015, Spring 2016.

\item[] \emph{Climate Change and Human Evolution}. ANTH 3491. I designed this upper level undergraduate course covering changes in global climate through evolutionary time and the impacts on evolution, with an emphasis on humans. (Previous course title: Evolutionary Impacts of Cenozoic Climate Change). George Washington University, Anthropology. Taught Spring 2016.

\item[] \emph{Public Understanding of Science}. HOMP 8302.  Graduate course in which students complete semester-long public service internships. Student projects target underserved Washington, DC-area public schools and general audiences at public museums with a goal of increasing scientific literacy and creating interest in scientific careers. George Washington University, Anthropology. Taught Spring 2016.

\item[] \emph{GIS and Remote Sensing for Archaeology and Paleontology}. ANT 391 / GRG 396. Teaching Assistant.  University of Texas at Austin, Anthropology. 2010.

\item[] \emph{Human Variation}. ANT 394C. Teaching Assistant. University of Texas at Austin, Anthropology. 2009.

\item[] \emph{Introduction to Physical Anthropology}. ANT 301. Teaching Assistant. University of Texas at Austin, Anthropology. 2006, 2007, 2010, 2011, 2012.
\end{description*}

\section*{Professional Preparation}
\begin{description*}

\item[] External Member. Evolution of Terrestrial Ecosystems Working Group. National Museum of Natural History. 2015 - Present.

\item[] Research Associate and Software Developer. PaleoCore Project. I am a key member of this NSF Funded project, which aims to create a data-standard for physical anthropology. I contributed heavily to the development of PaleoCore informatics tools for data sharing. 2012 - Present.

\item[] Participant. AnthroTree Workshop in Phylogenetic Methods. Amherst, MA. 2011.

\item[] Research Assistant. eSkeletons.org and eFossils.org. PI: John Kappelman. 2010, 2011.

\end{description*}

\end{document}
