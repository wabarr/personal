\documentclass{article}
\usepackage[top=1in, bottom=1in, left=1in, right=1in]{geometry}
\usepackage{mdwlist}
\usepackage{fancyhdr}
\usepackage{hyperref}
\pagestyle{fancy}
\pagenumbering{arabic}
\lhead{\itshape W. Andrew Barr - CV}
\chead{}
\rhead{\thepage}
\lfoot{}
\cfoot{Updated \today}
\rfoot{}

\thispagestyle{empty}

\begin{document}
\begin{center}
\noindent{\bfseries{\Huge W. Andrew Barr - Curriculum Vitae}}
\end{center}

\vspace{15pt}

\noindent\begin{minipage}{.60\textwidth}
\begin{flushleft}
Center for the Advanced Study of Human Paleobiology\\
Department of Anthropology\\
The George Washington University\\
\end{flushleft}
\end{minipage}
\begin{minipage}{.395\textwidth}
\begin{flushright}
800 22nd St NW, Suite 6000\\
Washington, DC 20052 \\
Email: wabarr@gmail.com\\
\end{flushright}
\end{minipage}


\noindent\rule[-2mm]{\textwidth}{1pt}

\section*{Education}
\begin{description*}
\item[] PhD - 2014 - University of Texas at Austin. Anthropology.
\item[] MA  - 2008 - University of Texas at Austin. Anthropology.
\item[] BS  - 2005 - Tulane University. Anthropology, French.
\end{description*}


\section*{Academic Appointments}
\begin{description*}
\item[] {\bfseries Postdoctoral Scientist}. Center for the Advanced Study of Human Paleobiology. Department of Anthropology. The George Washington University. Advisor: Bernard Wood. 2014 - Present.
\item[] {\bfseries Professorial Lecturer}. Department of Anthropology. The George Washington University. 2015.
\item[] {\bfseries Research Associate}. Department of Paleobiology.  Smithsonian National Museum of Natural History. 2014 - Present.
\end{description*}


\section*{Peer Reviewed Publications}

\begin{description*}
\item[] {\bfseries Accepted}
\item[] Barr, WA. Paleoenvironments of the Shungura Formation based on ecomorphology of the bovid astragalus. \emph{Journal of Human Evolution}.
\end{description*}


\begin{description*}
\item[] {\bfseries 2014}
\item[] Barr, WA. Functional Morphology of the Bovid Astragalus In Relation To Habitat: Controlling Phylogenetic Signal In Ecomorphology. \emph{Journal
of Morphology}. 275:1201-1216. \href{http://dx.doi.org/10.1002/jmor.20279}{doi:10.1002/jmor.20279}

\item[] Barr, WA and Scott, RS. Phylogenetic comparative methods complement discriminant function analysis in ecomorphology. \emph{American Journal
of Physical Anthropology}. 153:663-674. \href{http://dx.doi.org/10.1002/ajpa.22462}{doi:10.1002/ajpa.22462}

\item[] Scott, RS and Barr, WA. Ecomorphology and phylogenetic risk: implications for habitat reconstruction using fossil bovids.
\emph{Journal of Human Evolution}. 73:47-57. \href{http://dx.doi.org/10.1016/j.jhevol.2014.02.023}{doi:10.1016/j.jhevol.2014.02.023}

\end{description*}

\begin{description*}
\item[] {\bfseries 2010}
\item[] Reed, DN, and Barr, WA. A preliminary account of the rodents from Pleistocene levels at Grotte des Contrebandiers (Smuggler's Cave),
Morocco. \emph{Historical Biology}. 22:286-294. \href{http://dx.doi.org/10.1080/08912960903562192}{doi:10.1080/08912960903562192}
\end{description*}
\section*{Funding and Awards}

\begin{description*}
\item[] {\bfseries 2014}
\item[] Travel Grant - Paleoanthropology Society for meetings in Calgary. \$500
\end{description*}


\begin{description*}
\item[] {\bfseries 2013}
\item[] Named Continuing Fellowship - UT Austin Graduate School. \$29,000
\item[] Pollitzer Student Travel Award - American Association of Physical Anthropologists. \$500
\end{description*}


\begin{description*}
\item[] {\bfseries 2012}
\item[] Dissertation Fieldwork Grant - Wenner-Gren Foundation. \$13,317.
\end{description*}

\newpage
\begin{description*}
\item[] {\bfseries 2008}
\item[] Professional Development Award - Dept. of Anthropology, UT Austin. Also recieved in 2009-2011.
\end{description*}

\begin{description*}
\item[] {\bfseries 2007}
\item[] Graduate Research Fellowship - National Science Foundation. \$90,000

\item[] Liberal Arts Graduate Research Fellowship - UT Austin.

\item[] Student Prize - Texas Association of Biological Anthropologists.

\item[] David Bruton, Jr. Graduate Fellowship - UT Austin. Also received in 2008.
\end{description*}

\section*{Professional Preparation}
\begin{description*}

\item[] Research Associate \& Software Developer. PaleoCore Project. I am a key member of this NSF Funded project, which aims to create a data-standard for physical anthropology. I contributed heavily to the development of PaleoCore informatics tools for data sharing. 2012 - Present.

\item[] Observer. Evolution of Terrestrial Ecosystems Working Group. National Museum of Natural History. 2015

\item[] Participant. AnthroTree Workshop in Phylogenetic Methods. Amherst, MA. 2011.

\item[] Research Assistant. eSkeletons.org and eFossils.org. PI: John Kappelman. 2010, 2011.

\end{description*}

\section*{Teaching Experience}
\begin{description*}
\item[] \emph{Analytical Methods in Evolutionary Anthropology}. I designed and taught this course for all first-and second-year graduate students in the CASHP program at GWU. This course covered applied statistical methods (e.g, regression, ANOVA and related techniques, categorical data analysis, resampling approaches) and the R statistical programming language. Spring 2015

\item[] \emph{Introduction to Physical Anthropology}. Head Teaching Assistant. UT Austin. 2010, 2012.

\item[] \emph{GIS and Remote Sensing for Archaeology and Paleontology}. Teaching Assistant.  UT Austin. 2010.

\item[] \emph{Human Variation}. Teaching Assistant. UT Austin. 2009.

\item[] \emph{Introduction to Physical Anthropology}. Teaching Assistant. UT Austin. 2006, 2007, 2011.
\end{description*}

\section*{Fieldwork Experience}
\begin{description*}
\item[] Mille-Logya Research Project, Afar Region, Ethiopia. PI: Zeresenay Alemseged. I am a member of the scientific team conducting research to recover new fossil evidence of human evolution from this Plio-Pleistocene site. 2014.

\item[] Great Divide Basin Project, Wyoming. PI: Robert Anemone. Collected primate and mammalian fossils from Eocene sediments, and prospected for new localities. 2013.

\item[] Dikika Research Project, Afar Region, Ethiopia. PI: Zeresenay Alemseged. Surface collection of Plio- Pleistocene hominin and mammalian fossils. Managed GIS data collection with hand-held computers and high-precision GPS base station. 2010, 2012.

\item[] Dalquest Research Site, Big Bend Region, Texas. PI: E. Chris Kirk. Surface collected primate and mammalian fossils in the Devil's Graveyard Formation. (Eocene: Late Uintan). 2007, 2008, 2010.

\item[] Contrebandiers Cave, Temara, Morocco. PI: Harrold Dibble \& Utsav Schurmans. Excavated site preserving Middle Stone Age archaeology (Aterian) and hominin remains. Performed systematic analysis of rodent fauna. 2009.
\end{description*}
\section*{Scholarly Presentations}

\subsection*{Published Abstracts from Conference Presentations}

\begin{description*}
\item[] {\bfseries 2015}
\item[] Barr, WA and Dunn, RH. A method for analyzing complex joint surfaces in ecomorphology using slope rasters derived from Digital Elevation Models. American Association of Physical Anthropology.
\item[] Thompson, JC, McPherron SP, Bobe R, Barr, WA, Reed D, Wynn J, Marean CW, and Alemseged, Z. Taphonomy of fossils from the hominin-bearing deposits at Dikika, Ethiopia. Paleoanthropology Society.
\end{description*}

\begin{description*}
\item[] {\bfseries 2014}
\item[] Barr, WA. Paleoenvironments of the Hadar and Shungura Formations: Synthesizing multiple lines of evidence using bovid ecomorphology. American Association of Physical Anthropology.
\item[] Kemp, A and Barr, WA. Rates of homoplasy in the mammalian skeleton. American Association of Physical Anthropology.
\end{description*}

\begin{description*}
\item[] {\bfseries 2013}
\item[] Barr, WA. Ecomorphology of the bovid astragalus: body size, function, phylogeny and paleoenvironmental reconstruction. \emph{American Journal of Physical Anthropology}. 150:74.
\end{description*}
\begin{description*}
\item[] {\bfseries 2012}
\item[] Barr, WA. Ecomorphology in a phylogenetic statistical context: a case study using the bovid femur. \emph{American Journal of Physical Anthropology}. 147:90-91.

\item[] Scott, RS and Barr, WA. Ecomorphology and phylogeny among the Bovidae: implications for habitat reconstruction. \emph{American Journal of Physical Anthropology}.

\item[] Kappelman, JK, Keane P, Reed D, Tenbarge J, Witzel A, Barr WA, Nachman BA, Russo GA. eFossils.org: a collaborative website and community database for the study of human evolution. \emph{American Journal of Physical Anthropology}.
\end{description*}
\begin{description*}
\item[] {\bfseries 2011}
\item[] Reed DN, McPherron S, Barr WA, Alemseged Z, Bobe R, Geraads D, and Wynn J. A new GPS data collection methodology and data schema for integrating multiple project databases: examples from the Dikika Research Project geodatabase. \emph{American Journal of Physical Anthropology}. 144:249-250.
\end{description*}
\begin{description*}
\item[] {\bfseries 2009}
\item[] Barr, WA \& Reed, DN. Coping with taxonomic ambiguity and inter-observer variation in paleontological and paleoanthropological analyses. \emph{American Journal of Physical Anthropology}. 144:249-250.

\item[] Toborowsky, CJ \& Barr, WA \& Lewis, RJ. Does environmental unpredictability drive lemur life histories? \emph{American Journal of Physical Anthropology}.
\end{description*}
\begin{description*}
\item[] {\bfseries 2008}
\item[] Barr, WA. The effects of allometric scaling patterns on the template method for estimating dimorphism. \emph{American Journal of Physical Anthropology}.
\end{description*}
\subsection*{Scholarly Presentations Without Published Abstracts}

\begin{description*}
\item[] {\bfseries 2013}
\item[] Barr, WA, Brett Nachman, Liza Shapiro. The Academic Phylogeny of Physical Anthropology. Annual Meetings of the Texas Association for Biological Anthropologists. Austin, TX.

\item[] Reed D, Barr, WA, Urban T. TFree as in speech and beer: open source software solutions for spatial data management in physical anthropology. Annual Meetings of the Texas Association for Biological Anthropologists. Austin, TX.
\end{description*}
\begin{description*}
\item[] {\bfseries 2012}
\item[] Barr, WA. Refining hominin paleoenvironmental reconstructions using bovid ecomorphology: the role of phylogenetic comparative methods. Presentation at UT Austin Paleontology Brown-Bag Seminar Series.

\item[] Barr, WA. Refining hominin paleoenvironmental reconstructions with bovid ecomorphology. Presentation at UT Austin Informal Physical Anthropology Semininar series.
\end{description*}
\begin{description*}
\item[] {\bfseries 2011}
\item[] Barr, WA. Ecomorphology in a phylogenetic statistical context: a case study using the bovid femur. Annual Meetings of Texas Association for Biological Anthropologists. San Marcos, TX.
\end{description*}
\begin{description*}
\item[] {\bfseries 2010}
\item[] Barr, WA. Pattern or chaos? Exploring a null model of faunal turnover patterns. Annual Meetings of Texas Association for Biological Anthropologists. Waco, TX.

\item[] Barr, WA. Quantitative ecomorphology of mammalian dentitions: Refining a tool for reconstructing early hominin paleoenvironments. Presentation at UT Austin Informal Physical Anthropology Semininar series.
\end{description*}
\begin{description*}
\item[] {\bfseries 2008}
\item[] Barr, WA. Coping with taxonomic ambiguity and inter-observer variation in paleontological and paleoanthropological analyses. Annual Meetings of Texas Association for Biological Anthropologists. College Station, TX.
\end{description*}
\begin{description*}
\item[] {\bfseries 2007}
\item[] Barr, WA. The effects of allometric scaling patterns on the template method for estimating dimorphism. Annual Meetings of Texas Association for Biological Anthropologists. Austin, TX.
\end{description*}
\subsection*{Invited Talks and Guest Lectures}

\begin{description*}
\item[] {\bfseries 2013}
\item[] Data reshaping and advanced plotting with ggplot2. Guest Lecture for Denn\'{e} Reed in graduate-level Statistical Methods course at UT Austin.

\item[] Guest Lecture for E. Christopher Kirk on Eocene Primate evolution for the Introduction to Physical Anthropology course.
\end{description*}
\section*{Professional Service}
\begin{description*}
\item[] \href{http://www.physanthphylogeny.org}{Academic Phylogeny of Physical Anthropology} - In collaboration with Liza Shapiro and Brett Nachman, I created this website as a public resource that tracks academic lineages of Physical Anthropology PhDs. The site has had over 1700 user submissions. 2013.

\item[] Volunteer, Explore UT - University wide K-12 educational open house. Helped organize and run activity ``Leaping Lemurs of Madagascar'' on locomotion and conservation of lemurs. 2010, 2011.

\item[] Reviewer - University of Texas Liberal Arts Graduate Research Fellowship. Evaluated grant proposals from students competing for \$50,000 in grant funds. 2008.

\item[] Coordinator - Informal Physical Anthropology Seminar Series, UT Austin. Responsible for planning weekly seminars and recruiting speakers. 2008.
\end{description*}
\subsection*{Manuscript Reviews}
\begin{description*}
\item[] Journal of Human Evolution. 2014, 2015.
\item[] Methods in Ecology and Evolution. 2012.
\item[] International Journal of Primatology. 2014.
\item[] Manning Publications (book proposal review). 2012.
\end{description*}
\subsection*{Professional Memberships}
\begin{description*}
\item[] American Association of Physical Anthropologists
\item[] Paleoanthropology Society
\end{description*}
\end{document}
